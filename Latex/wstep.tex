\chapter*{Wstęp}

Planowanie to odpowiadanie na pytanie `Co powinniśmy stworzyć i kiedy?'.
Jednak aby odpowiedzieć na to pytanie powinniśmy zadać również pytania o estymację
(`Jak duże to jest?') oraz harmonogram (`Kiedy będzie skończone\text{?}` oraz `Ile będzie zrobione do tego czasu?'). Estymowanie i planowanie są bardzo istotne w sukcesie każdego projektu. Plany pomagają inwestorom podjąć decyzję. Na przykład możemy zacząć specyficzny projekt, jeżeli oszacujemy, że zajmie sześć miesięcy oraz będzie wymagać milion dolarów, ale odrzucimy go, jeżeli stwierdzimy, że zabierze nam dwa lata oraz 4 miliony dolarów.\cite{Cohen_2006}

\section*{Opis problemu}

W dzisiejszym świecie coraz więcej projektów jest tworzonych przez zespoły rozproszone. Zespół rozproszony to taki, którego członkowie realizują jeden projekt, cel i zadania w określonym czasie, pracując z różnych miejsc – kontynentów, krajów, miast, budynków czy biur. Zespoły rozproszone pracują jak tradycyjne zespoły zadaniowe, ale mają ze sobą na co dzień kontakt wirtualny i widują się sporadycznie.\cite{www_rozproszony} Przez co nie mogą się spotykać co sprint w jednym pomieszczeniu by oszacować zadania do wykonania w sprincie. Dlatego muszą się spotkać w wirtualnym pokoju w chmurze. Jednak muszą też swoje estymacje mieć zapisane w centralnym miejscu, gdzie mieści się ich projekt. Bardzo często tym miejscem jest np. Github.

\section*{Cel pracy}

Zadaniem jakie jakie postawiłem przed sobą jest stworzenie aplikacji umożliwiającej rozegranie planning pokera w czasie rzeczywistym w raz z rolami product ownera, scrum mastera oraz gracza by jak najwierniej zasymulować rozgrywkę w planning pokera. Dodatkowym celem jest synchronizowanie tego wszystkiego z aplikacją Github.

\section*{Zakres pracy}

Praca obejmowała opracowanie projektu aplikacji, implementację w frameworku ReactJS oraz wdrożenie wszystkiego powyższego w bazie danych Firebase oraz zhostowanie tego wszystkiego na hostingu Firebase`a. Dodatkowym zadaniem jest sprawdzenie użyteczności programu za pomocą ankiety.
