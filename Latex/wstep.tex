\chapter*{Wstęp}
Planowanie to odpowiadanie na pytanie „Co powinniśmy stworzyć i kiedy?”.
Jednak aby odpowiedzieć na to pytanie powinniśmy zadać również pytania o estymację
(„Jak duże to jest?”) oraz harmonogram („Kiedy będzie skończone\text{?}” oraz „Ile będzie zrobione do tego czasu?”).
Estymowanie i planowanie są bardzo istotne w sukcesie każdego projektu,
gdyż plany pomagają inwestorom podejmować decyzje.\cite{Cohen_2006}
\section*{Opis problemu}

W dzisiejszym świecie coraz więcej projektów jest tworzonych przez zespoły rozproszone.
Zespół rozproszony to taki, którego członkowie realizują jeden projekt,
cel i zadania w określonym czasie, pracując z różnych miejsc. 
Zespoły rozproszone pracują jak tradycyjne zespoły zadaniowe,
ale mają ze sobą na co dzień kontakt wirtualny i widują się sporadycznie.\cite{www_rozproszony}
Przez co mają problem ze spotkaniami na żywo.
Dlatego muszą się spotkać w wirtualnym pokoju a wnioski ze spotkania powinny być zapisane w centralnym miejscu projektu.
Bardzo często tym miejscem jest np. GitHub.\footnote{GitHub – hostingowy serwis internetowy przeznaczony dla projektów programistycznych wykorzystujących system kontroli wersji Git.}

\section*{Cel pracy}

Zadaniem jakie postawiłem przed sobą jest stworzenie aplikacji umożliwiającej rozegranie planning pokera w czasie rzeczywistym w raz z rolami product ownera,
scrum mastera oraz gracza by jak najwierniej zasymulować rozgrywkę w planning pokera.
Dodatkowym celem jest synchronizowanie wyników rozgrywki z aplikacją GitHub.

\section*{Zakres pracy}

Praca obejmuje opracowanie projektu aplikacji, implementację w frameworku ReactJS.\footnote{ReactJS – biblioteka języka programowania JavaScript, która wykorzystywana jest do tworzenia interfejsów graficznych aplikacji internetowych.}
oraz wdrożenie powyższego w bazie danych Firebase, a także zhostowanie tego wszystkiego na hostingu Firebase.\footnote{Firebase – to platforma do opracowywania aplikacji mobilnych i internetowych opracowana przez firmę Firebase, Inc.\ w 2011 roku, a następnie nabyta przez Google w 2014 roku.}
Dodatkowym zadaniem jest sprawdzenie użyteczności programu za pomocą ankiety.