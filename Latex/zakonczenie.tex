\chapter*{Zakończenie}

Celem pracy było zaprojektowanie oraz zaimplementowanie narzędzia wspomagającego szacowanie zadań metodą planning pokera.
Udało się zrealizować założenia projektu, jednak koncepcja często się zmieniała,
co wynikało z braku doświadczenia w tworzeniu aplikacji czasu rzeczywistego oraz poszukiwania odpowiednich rozwiązań.
Jednak aplikacja dzięki wykorzystaniu odpowiednich narzędzi została stworzona i przetestowana.

Uważam że usługa Firebase nadaje się świetnie do aplikacji wymagających asynchronicznej komunikacji.
Aplikacja została przetestowana na wszystkich przeglądarkach, a nawet na urządzeniach mobilnych.
Dzięki usłudze Firebase Hosting jest ona także dostępna w internecie.

Mimo, że podobne aplikacje już istnieją,
sądze że przedstawiony w niniejszej pracy program może znaleźć zainteresowanie wśród programistów.
Obserwując trendy na rynku pracy, aplikacje które pomagają w pracy zdalnej będą coraz bardziej poszukiwane.
Na koniec pragnę zauważyć, że mimo kilku zmian koncepcji, ostatecznie zaproponowana aplikacja spełnia w całości założone cele.
