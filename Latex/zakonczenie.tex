\chapter*{Zakończenie}

Celem pracy było zaprojektowanie oraz zaimplementowania narzędzia wspomagającego szacowanie zadań metodą planning pokera.
Udało się zrealizować założenia projektu, jednak koncepcja często się zmieniała,
co wynikało z braku doświadczenia w tworzeniu aplikacji czasu rzeczywistego oraz poszukiwania odpowiednich rozwiązań.
Jednak aplikacja dzięki wykorzystaniu odpowiednich narzędzi została stworzona i przetestowana.
Uważam że usługa firebase świetnie się nadaje do aplikacji wymagających asynchronicznej komunikacji.
Aplikacja została przetestowana na wszystkich przeglądarkach, a nawet na urządzeniach mobilnych.
Dzięki usłudze firebase hosting jest ona też dostępna w internecie.
Jako że podobne aplikacje już istnieją,
sądze że przedstawiony w niniejszej pracy program może znaleźć zainteresowanie wśród programistów.
Sądząc po trendach na rynku pracy, aplikację które pomagają w pracy zdalnej będą coraz bardziej popularne.
Co prawda autor nie sądzi by jego aplikacja zdobyła dużą popularność,
 jednak tego typu narzędzia będą coraz bardziej popularne dla wielu dzisiejszych i przyszłych firm.