\chapter*{Zakończenie}

Celem niniejszej pracy było zaprojektowanie oraz implementacja narzędzia wspomagającego zdalne szacowanie punktów historyjek użytkownika metodą planistycznego pokera (ang. Planning Poker).
Postawione cele i założenia projektu zostały w pełni zrealizowane.
Finalna wersja aplikacji poprzedzona była serią wielu implementacji prototypów w ramach wyboru optymalnego zestawu narzędzi programistycznych.
Wynika to z faktu, że technologie webowe są obecnie najdynamiczniej rozwijającą się gałęzią informatyki.
Ogromna liczba dostępnych tzw. frameworków do budowy aplikacji webowych, tak po stronie klienta - interfejsu użytkownika (tzw. frontend),
jak również wybór technologii po stronie serwera (w tym rozwiązania chmurowe)
wymusiła na autorze dogłębną ich analizę i przegląd pod kątem kompatybilności i możliwości użycia w projekcie.


Aplikacja webowa stworzona jako implementacja w ramach niniejszej pracy opiera się na modelu klient-serwer,
gdzie częścią serwerową (tzw. backend) zajmuje się usługa „chmurowa” firmy Google, tj. Firebase. Można zauważyć,
że usługa ta doskonale nadaje się do aplikacji wymagających komunikacji asynchronicznej,
co w przypadku narzędzia dla wielu użytkowników pracujących w tym samym czasie jest cechą nadrzędną.

Aby zapewnić użytkownikom, czyli tzw. stronie klienta (ang. client side) nowoczesny i funkcjonalny interfejs,
użyto tzw. frameworka React-Redux – na chwilę obecną silnego kandydata dla wielu projektów i Start-Upów ze względu na ogromne wsparcie społeczności Open-Source.

Hosting aplikacji został również oparty na jednej z usług „chmurowych” firmy Google – Firebase Hosting.
Znacznie ułatwia to zarządzanie zmianami w projekcie, dzięki wsparciu dla typowego obecnie procesu rozwoju oprogramowania w oparciu o system kontroli wersji (GIT).

Po serii testów funkcjonalnych na wiodących przeglądarkach internetowych, na systemach operacyjnych MS Windows,
Linux, a także Android zostały przeprowadzone również testy użyteczności.

Mimo, że podobne pod kątem wyglądu aplikacje już istnieją,
zaproponowana w ramach pracy aplikacja z cała pewnością może znaleźć zainteresowanie wśród członków zespołów projektowych,
zwłaszcza szczególnie ceniących sobie użycie sprawdzonych i nowoczesnych rozwiązań webowych,
dających swobodę użycia i oferujących bezpieczeństwo danych, tak ważne w dzisiejszych czasach.

Możliwość rozszerzenia najpopularniejszej na świecie platformy do zarządzania projektami open-source – GitHub o funkcję estymacji historyjek / zadań (ang. user stories / issues),
w oparciu o API i metody autentykacji oraz autoryzacji tejże platformy,
pozwala na natychmiastowe przejście całego zespołu projektowego na interfejs stworzonej aplikacji celem kontynuacji procesu estymacji.

Pozwala to wyeliminować szereg zbędnych czynności organizacyjnych i konfiguracyjnych oraz buduje w użytkownikach poczucie bezpieczeństwa szczególnie,
iż nie jest wymagane tworzenie kolejnych kont użytkownika a dane projektowe są pobierane z platformy GitHub i przechowywane na zabezpieczonej platformie Firebase.
Obserwując obecne trendy na rynku pracy\cite{www_rozproszony} należy przypuszczać, że aplikacje wspierające pracę zdalną będą coraz bardziej poszukiwane.