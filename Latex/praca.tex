\documentclass[inzynier,druk]{dyplom}
\usepackage[utf8]{inputenc}
\usepackage{hyperref}
\usepackage{booktabs}
%%
\usepackage[toc]{appendix}
\renewcommand{\appendixtocname}{Dodatki}
\renewcommand{\appendixpagename}{Dodatki}

% pakiet do sk?adu listingów w razie potrzeby można odblokować możliwość numerowania linii lub zmienił wielkość czcionki w listingu
\usepackage{minted}
\setminted{breaklines,
frame=lines,
framesep=5mm,
baselinestretch=1.1,
fontsize=\small,
%linenos
}

% nowe otoczenie do składania listingów
\usepackage{float}
\newfloat{listing}{htp}{lop}
\floatname{listing}{Listing}
\let\counterwithout\relax
\let\counterwithin\relax
\usepackage{chngcntr}
\counterwithin{listing}{chapter}

% patch wyr?wnuj?cy spis listing?w do lewego marginesu
%https://tex.stackexchange.com/questions/58469/why-are-listof-and-listoffigures-styled-differently
\makeatletter
\renewcommand*{\listof}[2]{%
  \@ifundefined{ext@#1}{\float@error{#1}}{%
  \expandafter\let\csname l@#1\endcsname \l@figure% <- use layout of figure
    \float@listhead{#2}%
    \begingroup
      \setlength\parskip{0pt plus 1pt}%               % <- or drop this line completely
        \@starttoc{\@nameuse{ext@#1}}%
    \endgroup}}
\makeatother

\usepackage{url}
\usepackage{lipsum}
\usepackage{pdflscape}
\usepackage{multirow}
\usepackage{makecell}
\usepackage{geometry}

% Dane o pracy
\author{Karol Kowalski}
\title{Aplikacja wspomagająca zdalne szacowanie historyjek użytkownika metodą
Planistycznego Pokera}
\titlen{Application supporting remote user stories estimation
using Planning Poker method.}
\promotor{dr \ hab. \ inż. \ prof. PWr. Trawiński Bogdan}
%\konsultant{dr hab. in?. Kazimerz Kabacki}
\wydzial{Wydział Informatyki i Zarządzania}
\kierunek{Informatyka}
\krotkiestreszczenie{Praca przedstawia analizę problemu estymacji zadań
w projektach prowadzonych metodykami zwinnymi. Zawiera opis projektu aplikacji
umożliwiającej przeprowadzenie rozgrywki planning pokera online,
stworzonej przy użyciu nowoczesnych narzędzi webowych: ReactJS i Firebase.}
\slowakluczowe{metodyki zwinne, estymacja zadań, historyjki użytkownika, planning poker}

\begin{document}

\maketitle

\tableofcontents

\listoffigures

\listof{listing}{Spis listingów}

\listoftables

% --- Strona ze streszczeniem i abstraktem ------------------------------------------------------------------
\chapter*{Streszczenie} % po polsku
Niniejsza praca zawiera omówienie wybranych metod zwinnych zarządzania projektami,
ze szczególnym uwzględnienie technik wykorzystywanych
do estymacji zadań projektowych w tych metodykach.
Spośród znanych autorowi technik, najbardziej rozpowszechnioną jest tzw. Planning Poker.
Można znaleźć wiele aplikacji wspomagających proces estymacji przy wykorzystaniu
tej techniki, jednakże autor nie znalazł wśród nich żadnej, która pozwalałaby
na integrację z popularnym narzędziem GitHub.

Celem pracy było zaprojektowanie oraz wykonanie aplikacji wspomagającej grę
Planning Poker. Jej potencjalnymi użytkownikami są zespoły programistów
pracujących w trybie zdalnym, które wykorzystują narzędzie GitHub w codziennej pracy.

W ramach pracy autor stworzył aplikację typu Single-page opartą o architekturę Flux
oraz wykorzystującą usługę Firebase jako tzw. backend.
Wyróżniającą cechą aplikacji jest możliwość importu historyjek
(tzw. issues) z GitHub oraz zapisu rezultatów sesji planningowych za pomocą
odpowiedniego etykietowania historyjek w GitHub.

Użyteczność aplikacji została zweryfikowana w oparciu o wyniki ankiety,
które zostały zawarte w niniejszej pracy.


% Kilka sztuczek, ?eby:
% - Abstract pojawi? si? na tej samej stronie co Streszczenie
% - Abstract nie pojawi? si? w spisie tre?ci
\addtocontents{toc}{\protect\setcounter{tocdepth}{-1}}
\begingroup
\renewcommand{\cleardoublepage}{}
\renewcommand{\clearpage}{}
\chapter*{Abstract} % ...i to samo po angielsku
The thesis describes a few agile project management methods with the emphasis
put on techniques used for user stories estimation.
The well-established technique is Planning Poker.
There are many application that support this technique, but author of the thesis
could find no application that would offer any kind of integration with
GitHub.

The main goal of this thesis was to design and develop a web application to play
Planning Poker online. The target of the application are remote teams of software
developers that use GitHub in everyday work.

The author developed a Single-page application based on the Flux architecture
and using  the Firebase service as a backend.
The \textit{killer feature} of the application is the integration with
GitHub that allows for importing and filtering issues and saving the results of
estimation sessions as issue labels.

The usability of application was verified based on a survey given to
application users. The survey results are included in the thesis.
\endgroup
\addtocontents{toc}{\protect\setcounter{tocdepth}{2}}
% --- Koniec strony ze streszczeniem i abstraktem -----------------------------------------------------------

\chapter*{Wstęp}
Planowanie to odpowiadanie na pytanie „Co powinniśmy stworzyć i kiedy?”.
Jednak, aby odpowiedzieć na to pytanie powinniśmy zadać również pytania o estymację
(„Jak duże to jest?”) oraz harmonogram („Kiedy będzie skończone\text{?}” oraz „Ile będzie zrobione do tego czasu?”).
Estymowanie i planowanie są bardzo istotne w sukcesie każdego projektu,
gdyż plany pomagają inwestorom podejmować decyzje.\cite{Cohen_2006}
\section*{Opis problemu}

W dzisiejszym świecie coraz więcej projektów jest tworzonych przez zespoły rozproszone. Zespołem rozproszonym nazywamy taki zespół, którego uczestnicy realizują jeden projekt, zadania i cel w tym samym czasie, pracując z różnych miejsc. Takie zespoły pracują jak zwykłe zespoły zadaniowe, ale kontaktują się ze sobą na co dzień wirtualnie i widują się sporadycznie.\cite{www_rozproszony}
Przez co mają problem ze spotkaniami na żywo.
Dlatego muszą się spotkać w wirtualnym pokoju a wnioski ze spotkania powinny być zapisane w centralnym miejscu projektu.
Bardzo często tym miejscem jest np. GitHub.\footnote{GitHub – hostingowy serwis internetowy przeznaczony dla projektów programistycznych wykorzystujących system kontroli wersji Git.}

\section*{Cel pracy}

Zadaniem jakie postawiłem przed sobą jest stworzenie aplikacji umożliwiającej rozegranie planning pokera w czasie rzeczywistym w raz z rolami product ownera,
scrum mastera oraz gracza, by jak najwierniej zasymulować rozgrywkę w planning pokera.
Dodatkowym celem jest synchronizowanie wyników rozgrywki z aplikacją GitHub.

\section*{Zakres pracy}

Praca obejmuje opracowanie projektu aplikacji, implementację w frameworku ReactJS.\footnote{ReactJS – biblioteka języka programowania JavaScript, którą wykorzystuje się do tworzenia graficznych interfejsów dla aplikacji internetowych.}
oraz wdrożenie powyższego w bazie danych Firebase, a także zhostowanie tego wszystkiego na hostingu Firebase.\footnote{Firebase – to platforma do opracowywania aplikacji mobilnych i internetowych opracowana przez firmę Firebase, Inc.\ w 2011 roku, a następnie nabyta przez Google w 2014 roku.}
Dodatkowym zadaniem jest sprawdzenie użyteczności programu za pomocą ankiety.


% !TeX program = latexmk
% !TeX spellcheck = pl_PL
% !TeX root = example.tex

\chapter{Wprowadzenie}

\begin{quote}
Projekt to tymczasowa działalność podejmowana w celu wytworzenia unikatowego wyrobu,
dostarczenia unikatowej usługi lub otrzymania unikatowego rezultatu.
~\cite{PMI_2000}
\end{quote}

Powyższe zdanie to najbardziej znana definicja projektu, stworzona przez
Project Management Institute (PMI).
PMI powstał w 1969 roku w Pensylwanii w USA jako stowarzyszenie non profit
zrzeszające profesjonalistów w dziedzinie zarządzania projektami.

Już w starożytnym Egipcie istniały metody zarządzania skomplikowanym przedsięwzięciem.
Jednym z takich przedsięwzięć była np. budowa piramid.
Było to olbrzymie wyzwanie, które wymagało wiedzy zarówno planistycznej jak i logistycznej.
W ubiegłym stuleciu, w latach 50-tych, stosowano podejścia zwane obecnie współczesnymi
technikami zarządzania projektami. Weźmy np. projekt systemu rakiet balistycznych Polaris.
Okazał się on swoistym koszmarem technicznym i administracyjnym.
Nad projektem pracowała olbrzymia ilość zespołów badawczych, projektowych i produkcyjnych.
Dla udokumentowania wszystkich działań zużyto tony papieru,
a samo zarządzanie projektami zaczęto uznawać za dziedzinę bardzo skomplikowaną,
niedostępną, opartą na wiedzy specjalistów.
\cite{Stanley_2013}

\section{Najpopularniejsze metody zarządzania projektami}
Przy dokonywaniu wyboru metodyki zarządzania projektem należy przeprowadzić
adekwatną analizę w celu doboru odpowiedniego podejścia,
gdyż każda z metodyk posiada wady i zalety.
Wyróżniamy dwa podejścia (klasyczne i zwinne), które różnią się dość mocno między sobą
w kilku płaszczyznach (przekrojach) takich jak:
odpowiedzialność za produkt, rola menedżera w zespole, istota prac wstępnych,
zdefiniowanie produktu czy odpowiedź zwrotna użytkowników.
Co uwzględniono w tabeli \ref{tabela:roznice}.

% Tabela. Nazwa tabeli u góry.
\begin{table}
\centering\caption{Różnice między podejściem klasycznym a zwinnym\label{tabela:roznice}}
\begin{tabular}{ p{0.25\textwidth} p{0.3\textwidth}  p{0.3\textwidth} }% wyrównanie kolumn tabeli -> l c r - do lewej, środka, do prawej
\toprule
\textbf{Płaszczyzna} &\textbf{ Podejście klasyczne} & \textbf{Podejście zwinne} \\
\midrule
\textbf{Odpowiedzialność za produkt}
 & Podzielona między marketera, menadżera produktu i menadżera projektu.
 & Istnieje tylko jeden właściciel produktu. \\
\midrule
\textbf{Rola menedżera w zespole}
& Oddzielony od zespołów deweloperskich.
& Jest członkiem zespołu i ściśle z nim współpracuje.\\
\midrule
\textbf{Istota prac wstępnych}
& Przeprowadzane są szczegółowe badania rynku, planowanie produktu i analizy biznesowe.
& Ograniczają się do stworzenia wizji, która ogólnie opisuje wygląd i działanie produktu.\\
\midrule
\textbf{Zdefiniowanie produktu}
& Wymagania są określane i zatwierdzane w początkowej fazie.
& Produkt odkrywany jest stopniowo, a wymagania krystalizują się w trakcie trwania projektu.\\
\midrule
\textbf{Odpowiedź zwrotna}
& Dostępna po wypuszczeniu produktu na rynek.
& Wczesna i częsta odpowiedź zwrotna po małych wdrożeniach.\\
\bottomrule
\end{tabular}
\end{table}
\newpage

\section{Podejście klasyczne}

Podejście klasyczne, reprezentowane przez PMBoK (Kompendium wiedzy o zarządzaniu projektami) lub metodykę PRINCE
(kompleksowa metoda zarządzania projektami, zalicza się ją do podejścia klasycznego) oraz jej następcę PRINCE2,
ma na celu wytworzenie kompletnego produktu przy uprzednim, dokładnym określeniu jego cech.
Takie podejście charakteryzuje się ogromnym formalizmem, weźmy na przykład dokonywanie zmian,
które wiąże się z wypełnianiem dokumentów RFC (z ang. Request for Change) - prośby o zmianę.
Każda z nich wiąże się z oczekiwaniem aż zostanie przeanalizowana i zatwierdzona bądź odrzucona.
Dodatkowo osoby odpowiedzialne zwykle nie pracują wraz z zespołem wykonawczym,
w związku z czym często występują bariery i opóźnienia w komunikacji.
\cite{www_tradycyjne_projekty}

\section{Podejście zwinne}

Tzw. podejście zwinne (ang. agile) w zarządzaniu projektami ukształtowało się
wraz z wielokrotnie powatarzającymi się niepowodzeniami projektów prowadzonych
metodami klasycznymi. Zwłaszcza w obszarze projektów informatycznych, których
poziom skomplikowania jest bardzo wysoki ze względu na to, iż muszą one modelować
wycinek rzeczywistości charakteryzujący się ogromną liczbą zmiennych.

Podejście zwinne ukierunkowane jest na zespół, który w pełni odpowiada za wykonanie swojej części
zadania i stopniowo dostosowuje je do potrzeb przyszłych użytkowników.
Przykładowe metodyki zwinne to m.in.: Scrum, Lean, Kanban, XP (ang. eXtreme Programming).

Każda z wyżej wymienionych metodyk posiada swoje charakterystyczne cechy,
nie da się jednak stwierdzić, że któraś jest lepsza od pozostałych.
Najważniejsze jest dobranie odpowiedniej metodyki do realiów pracy
i prawidłowa adaptacja względem realiów biznesowych,
gdyż ścisłe stosowanie wszystkich praktyk może być nadmiernie pracochłonne
w zastosowaniu do małych projektów.

Nazywamy je \textit{metodykami} a nie \textit{metodami}, ponieważ stanowią one
zwykle zestaw sugestii i dobrych praktyk, a nie zaś sztywną listę reguł.
Dlatego w praktyce zespoły często nie wykorzystują jednej metodyki, a opierają się na kilku,
dostosowując je do własnych potrzeb.
Przykładem może być tutaj niedawno powstały Scrum-ban, który jest połączeniem
dobrych praktyk zaczerpniętych z metodyk Scrum oraz Kanban.
~\cite{Wolf_2012}


\section{Manifest Agile}

Manifest Agile powstał w 2001 roku, ale nie jest to sam początek tego ruchu zwinnego oprogramowania.
Już wcześniej istniały pewne metodyki, jak również istniały podstawy teoretyczne
dla wprowadzenia takich rozwiązań. Jeśli chodzi o podstawy teoretyczne,
to trzeba zwrócić uwagę przede wszystkim na 3 kwestie:
\begin{itemize}
	\item kwestię podejścia systemowego i szkoły systemowej (czyli lata 70-te XX w.),
	która dostarczyła dość znaczącej wiedzy pozwalającej na współczesne zarządzanie projektami;
	\item zarządzanie jakością, koncepcje, metody zarządzania jakością,
	które są wykorzystywane bardzo mocno w metodykach zwinnych;
	\item zarządzanie wiedzą, czyli chociażby Takeuchi i Nonaka,
	którzy jako pierwsi wspomnieli o idei młyna (w artykule \textit{The New Product Development Game},
	opublikowanym w \textit{Harvard Business Review} w 1986r.), skąd wzięła się później metodyka \textit{Scrum}.
\end{itemize}

W latach 90-tych zaobserwowano znaczące skomplikowanie procesu tworzenia oprogramowania.
Projekty dotychczas zarządzane klasycznie okazały się zbyt mało elastyczne,
nie pozwalały na wystarczająco szybkie reagowanie na zmiany wymagań biznesowych.
Zdecydowano więc, że trzeba jakoś zmodyfikować sposób, w jaki tworzymy oprogramowanie,
aby odpowiadać na potrzeby klientów oraz potrzeby rynku wystarczająco szybko.

Pierwsze próby podjęto już na początku lat 90-tych XX wieku. Zostały one uwieńczone publikacją w 1995r.
metodyki Scrum, opisu sposobu w jaki można stosować tą metodykę.
Rok później pojawiła się metodyka XP (ang. eXtreme Programming),
a więc już w połowie lat 90-tych mieliśmy metodyki zwinne, które stosowane były
najpierw w ograniczonym zakresie, potem coraz szerzej.

Również poszczególne organizacje i przedsiębiorstwa zaczęły tworzyć swoje odmiany tych metod.
Tak więc dziś mamy całe bogactwo metodyk związanych z zarządzaniem zwinnym w projektach.

Aby określić co jest najważniejsze w zwinnym zarządzaniu projektami, w 2001r powstał manifest Agile
(ang. Agile Manifesto) przedstawiony na \ref{rys:agile}.

\begin{figure}
	\centering\includegraphics[width=.6\textwidth]{img/agile}
	\caption{Manifesto for Agile Software Dev.[www.medium.com]}.
	\label{rys:agile}
\end{figure}

Manifest przedstawia pewien system wartości, definiuje rzeczy najistotniejsze w zarządzaniu projektem.
Mamy więc cztery porównania:

\begin{itemize}
	\item Autorzy Manifestu stwierdzili, że ludzie i interakcje między nimi są ważniejsi,
	niż procesy i narzędzia. To nie znaczy, że procesy i narzędzia nie są istotne,
	ale są mniej ważne. Trzeba położyć większy nacisk na ludzi i na interakcje pomiędzy nimi.
	To powoduje bardziej nieformalną komunikację, jej przyspieszenie, co się przekłada
	na szybsze i bardziej elastyczne realizowanie zadań i umożliwia bardziej
	elastyczne realizowanie tych zadań, kiedy zmieniają się warunki.
	\item Druga zasadą jest orientacja bardziej na działające oprogramowanie niż na dokumentację.
	Jeśli zerkniemy do starych wersji oprogramowania z lat 80-tych, 90-tych,
	do każdego programu dodawana była gruba instrukcja.
	Dzisiaj już o tym zapomnieliśmy. Dzisiaj wiele aplikacji nie ma w ogóle nawet instrukcji
	- mówimy, że działają intuicyjnie (przynajmniej powinny).
	Dzięki temu, że orientujemy się na realizację tych najważniejszych efektów w projekcie,
	możemy lepiej wykorzystać zasoby, możemy szybciej osiągnąć te efekty, a rzeczy mniej ważne,
	mniej istotne, takie jak szczegółowa dokumentacja (jakaś dokumentacja przecież musi być),
	możemy odłożyć, możemy przeznaczyć dla nich mniejsze zasoby.
	\item Również jeśli chodzi o współpracę z klientem, w metodykach zwinnych
	proponuje się zmianę podejścia. Zamiast negocjować szczegółowo umowy,
	budujemy współpracę z tym klientem dlatego, że nie jesteśmy w stanie z góry przewidzieć,
	jaki będzie, tak do końca, zakres naszego projektu, co w tym projekcie zrealizujemy,
	co będzie potrzebne za rok, kiedy nasz produkt będzie prawie gotowy.
	Czy te wymagania się nie zmienią wielokrotnie, biorąc pod uwagę szybkość zmiany technologii,
	potrzeb, oczekiwań klientów, szybkość zmian na rynku. Zatem klient powinien być blisko,
	powinien dostarczać bieżące informacje o swoich potrzebach, a w kontrakcie zawieramy tylko te informacje,
	które są najważniejsze.
	\item I w końcu reagowanie na zmiany zamiast szczegółowego planowania.
	Oczywiście planowanie występuje w metodykach zwinnych, ale jest ono ograniczone tylko do tego,
	żeby dało się zarządzać takim projektem. Natomiast przede wszystkim orientujemy się na
	reagowanie na zmiany: zmiany potrzeb klienta, zmiany na rynku. Na dostosowanie naszego projektu,
	w kolejnych iteracjach, do tego, czego klient oczekuje.
\end{itemize}

Czasem niektórzy mówią, może żartobliwie, ale nieraz całkiem serio, że jeżeli czegoś nie zaplanowali,
to stosowali właśnie Agile. Nic bardziej błędnego: w Agile każda iteracja jest planowana,
w każdym dniu planujemy swoją pracę, stosujemy inne metody, rzadko stosujemy harmonogram Gantta,
ale także planujemy te działania. Zatem taki polski Agile („polnische Agile”, jak niektórzy mówią),
to przykład niewłaściwego zarządzania przedsięwzięciami i raczej nie należy się tym chwalić.
Warto jednak zauważyć, że nie do każdego projektu możemy zastosować metodyki zwinne.
One się lepiej sprawdzają wtedy, kiedy mamy:
\begin{itemize}
	\item bardzo krótkie, napięte terminy;
	\item projekty mają charakter unikatowy;
	\item są skomplikowane.
\end{itemize}

Mamy do zrealizowania coś nowego, nieoczekiwanego i mało czasu.
Wtedy ta metodyka zwinna rzeczywiście jest bardziej uzasadniona niż metodyki klasyczne.
Stosowanie metodyk zwinnych, szczególnie żądanie tej interakcji między pracownikami,
ogranicza nam wielkość zespołu, a więc ogranicza nam wielkość projektu.
Generalnie metodyki zwinne stosujemy:
\begin{itemize}
	\item w małych i średnich projektach, rzadziej w projektach dużych;
	\item konieczne jest aby w metodyce zwinnej dostępny był dla nas klient,
	klient musi się na bieżąco kontaktować z nami i mówić czego potrzebuje,
	jakie są jego oczekiwania, czy jest zadowolony z tego, co uzyskuje w poszczególnych iteracjach;
	\item tematyka projektu musi być taka, aby klient z każdej iteracji miał jakąś wartość,
	bowiem staramy się często wypuszczać oprogramowanie, często wprowadzać nowe jego wersje,
	ale to powoduje, że ta nowa wersja musi dostarczyć jakąś wartość dla klienta.
\end{itemize}

W przypadku oprogramowania jest to oczywiste. W przypadku, kiedy budujemy jakiś budynek,
być może Agile wtedy nie jest aż tak przydatny.
Trzeba się zastanowić, czy możemy zastosować całą metodykę Agile,
czy jak współcześnie w wielu projektach, zastosować ją tylko w odniesieniu
do wybranych modułów projektu, tam, gdzie rzeczywiście ma ona zastosowanie.

Więcej informacji na temat metodyk zwinnych można znaleźć w \cite{Cohen_2006}.


\input{Narzedzia}

\input{Rozproszone_zespoly}

\chapter{Założenia projektowe}

Tematem pracy jest stworzenie aplikacji wspomagającej szacowanie historyjek metodą Planning Poker.
Aplikacja pozwoli na przeprowadzenie gry, której backlog będzie pochodził z serwisu Github,
a wyniki gry będą zapisywane jako etykiety opisujące poszczególne zadania w Github.

Aplikacja będzie synchronizowana z Githubem,
czyli że jeżeli jakieś zadanie będzie z niego usunięte, usunięte zostanie również
z aplikacji.

Gra będzie posiadała także możliwości zamknięcia danego zadanie oraz jego edycji.

Inną planowaną funkcją jest możliwość tworzenia własnej skali estymacji.

Gracze będą dodawani do gry przez dzielenie się linkiem,
jednakże właściciel będzie mógł gracza wyrzucić z gry.

\section{Wymagania funkcjonalne}

\begin{itemize}
    \item Logowanie przez Github
    \item Logowanie jako gość (anonimowe)
    \item Pobieranie historyjek z Github
    \item Synchronizacja z Github
    \item Możliwość tworzenia list z issues Github
    \item Możliwość tworzenia gry
    \item Możliwość zapraszania graczy
    \item Możliwość głosowania przez graczy
    \item Możliwość tworzenia własnej skali estymacji
    \item Możliwość komentowania każdej historyjki podczas gry
    \item Możliwość automatycznego przeliczania wyników
    \item Możliwość wyrzucenia gracza z gry przez właściciela
    \item Możliwość eksportu wyników gry w postaci historyjek do Github
\end{itemize}

\section{Wymagania niefunkcjonalne}

\begin{itemize}
    \item Oprogramowanie musi być responsywne i proste.
    \item Aplikacja powinna się uruchamiać w przeglądarce internetowej.
    \item Aplikacja powinna działać w najnowszych wersjach przeglądarek.
    \item Aplikacja powinna działać w czasie rzeczywistym.
    \item Serwer powinien obsłużyć co najmniej 200 połączeń użytkowników.
    \item Zapewnienie bezpieczeństwa danych.
\end{itemize}

\section{Sposób komunikacji}

Aby aplikacja mogła komunikować się w czasie rzeczywistym niezbędne jest wykorzystanie
technologii Websockets, dzięki której możliwa będzie dwukierunkowa asynchroniczna komunikacja.

\section{Wykorzystanie technologie}

Aplikacja zostanie napisana w języku EcmaScript 6, kompilowanego do JavaScript
z wykorzystaniem SCSS, JSX i HTML5.

Wybrany zestaw technologii:
\begin{itemize}
    \item Node.js - środowisko uruchomieniowe do tworzenia aplikacji
    \item Yarn - Manadżer pakietów
    \item React -  biblioteka do tworzenia interfejsu użytkownika
    \item Redux - biblioteka wspomagająca zarządzanie stanem aplikacji
    \item Bootstrap - framework SCSS do budowania responsywnych stron internetowych
    \item Webpack - narzędzie do opakowywania, kompilowania i minimalizowania kodu
    \item Firebase - platforma służąca jako backend aplikacji
\end{itemize}


\chapter{Projekt aplikacji}
W niniejszym rozdziale autor umieścił projekt aplikacji oraz bazy danych.
\section{Diagram przypadków użycia}
\begin{figure}[H]
	\centering\includegraphics[width=0.98\textwidth]{img/UseCase.pdf}
	\caption{Diagram przypadków użycia}.\label{rys:useCase}
\end{figure}
\textbf{Product owner chce się zalogować}
\begin{enumerate}
    \item System prosi użytkownika o zalogowanie.
    \item Użytkownik naciska przycisk.
    \item Dostawca usługi logowania loguje użytkownika lub prosi o dane logowana.
    \item Product owner\footnote{Product owner jest moderatorem gry oraz osobą z dostępem do repozytorium projektu} zostaje zalogowany do systemu.
    \item Product owner przegląda listę swoich projektów.
\end{enumerate}
\textbf{Product owner chce stworzyć listę}
\begin{enumerate}
    \item Product owner loguje się do systemu.
    \item Product owner wybiera projekt dla którego chce stworzyć listę.
    \item Product owner wybiera elementy do nowej listy z `Issues' Github'a.
    \item Product owner naciska przycisk do tworzenia listy.
    \item Product owner nadaje listę nazwę i potwierdza wybór historyjek.
\end{enumerate}
\textbf{Product owner chce stworzyć grę}
\begin{enumerate}
    \item Product owner loguje się do systemu.
    \item Product owner wybiera projekt dla którego chce stworzyć grę.
    \item Jeżeli w projekcie jest stworzona jakakolwiek lista system prosi product ownera o wypełnienie formularza,
    w przeciwnym razie system odsyła użytkownika do listy issue's GitHub w celu stworzenia listy.
    \item Użytkownik potwierdza swoje ustawienia.
    \item System tworzy grę i przekierowuje użytkownika do listy gier projektu.
\end{enumerate}
\textbf{Product owner chce przeprowadzić grę}
\begin{enumerate}
    \item Product owner loguje się do systemu.
    \item Product owner tworzy grę.
    \item Product owner wchodzi do gry.
    \item Product owner rozsyła link do gry graczom.
    \item Gracze głosują nad historyjką.
    \item Product owner zmienia historyjki nadając tempo grze.
\end{enumerate}
\textbf{Product owner chce zmienić ustawienia gry}
\begin{enumerate}
    \item Product owner loguje się do systemu.
    \item Product owner wybiera projekt.
    \item Product owner wybiera grę.
    \item Product owner wchodzi do gry.
    \item Product owner naciska przycisk ustawienia.
    \item Product owner przechodzi do ekranu ustawień gry.
    \item Product owner zmienia ustawienia.
    \item Product owner zatwierdza ustawienia.
    \item System zapisuje ustawienia w bazie.
\end{enumerate}
\textbf{Product owner chce usunąć gracza z gry}
\begin{enumerate}
    \item Product owner loguje się do systemu.
    \item Product owner wybiera projekt.
    \item Product owner wybiera grę.
    \item Product owner wchodzi do gry.
    \item Product owner naciska przycisk `Gracze'.
    \item System wyświetla product ownerowi listę graczy.
    \item Product owner naciska przycisk usunięcia przy nazwie gracza.
    \item System usuwa gracza oraz wszystkie jego oceny z gry ponownie obliczając punktację.
\end{enumerate}
\textbf{Product owner chce wyeksportować punktację}
\begin{enumerate}
    \item Product owner loguje się do systemu.
    \item Product owner wybiera projekt.
    \item Product owner wybiera grę.
    \item Product owner naciska przycisk eksport obok przycisku usuwania gry.
    \item System wyświetla product ownerowi listę historyjek wraz z ich punktacją.
    \item Product owner zatwierdza punktację.
    \item System przenosi punktację historyjek w postaci etykiet do Github'a.
\end{enumerate}
\textbf{Product owner chce usunąć grę}
\begin{enumerate}
    \item Product owner loguje się do systemu.
    \item Product owner wybiera projekt.
    \item Product owner wybiera grę.
    \item Product owner naciska przycisk usuń.
    \item System usuwa grę z bazy.
    \item Gra znika z listy gier.
\end{enumerate}
\section{Interfejs}
Aby ułatwić implementację systemu, autor zamodelował ekrany aplikacji w postaci
tzw.\ makiet (ang.\ mockups):
\begin{itemize}
    \item Ekranu logowania
    \item Ekrany listy projektów
    \item Ekrany projektu wraz z bocznym menu (ekran przedstawiający issues)
    \item Ekran formularza tworzenia gry
    \item Ekran listy gier
    \item Ekran gry
\end{itemize}
Prototyp interfejsu użytkownika został stworzony dzięki narzędziu Visual Paradigm.
\begin{figure}[H]
	\centering\includegraphics[width=.7\textwidth]{img/LoginScreen.png}
	\caption{Ekran Logowania}.\label{rys:loginScreen}
\end{figure}
\begin{figure}[H]
	\centering\includegraphics[width=.7\textwidth]{img/RepositoriesScreen.png}
	\caption{Ekran główny aplikacji z repozytoriami}.\label{rys:RepositoriesScreen}
\end{figure}
\begin{figure}[H]
	\centering\includegraphics[width=.7\textwidth]{img/IssuesScreen.png}
	\caption{Główny ekran projektu}.\label{rys:IssuesScreen}
\end{figure}
\begin{figure}[H]
	\centering\includegraphics[width=.7\textwidth]{img/gameCreate.png}
	\caption{Formularz tworzenia gry}.\label{rys:gameCreate}
\end{figure}
\begin{figure}[H]
	\centering\includegraphics[width=.7\textwidth]{img/GamesList.png}
	\caption{Ekran gier}.\label{rys:GamesList}
\end{figure}
\begin{figure}[H]
	\centering\includegraphics[width=.7\textwidth]{img/GameScreen.png}
	\caption{Panel gry}.\label{rys:GameScreen}
\end{figure}

Na rysunku:~\ref{rys:ScreensDiagram} przedstawiono diagram przepływu pomiędzy
zaprezentowanymi powyżej makietami ekranów aplikacji.

\begin{figure}[H]
	\centering\includegraphics[width=\textwidth]{img/ScreensDiagram.png}
	\caption{Diagram przepływu między ekranami aplikacji}.\label{rys:ScreensDiagram}
\end{figure}

\section{Baza danych}

\begin{figure}[H]
	\centering\includegraphics[width=\textwidth]{img/ClassDiagram.png}
	\caption{Diagram klas, przedstawiający koncepcję bazy danych}.\label{rys:ClassDiagram}
\end{figure}

Na rysunku~\ref{rys:ClassDiagram} przedstawiono model danych w aplikacji.
W tym miejscu wartym odnotowania jest fakt, iż autor zdecydował się
na zastosowanie hierarchicznej bazy danych Firestore, zamiast którejś z najczęściej
wybieranych relacyjnych baz danych.
Więcej o Firestore znajdziemy w rozdziale o architekturze aplikacji.

Ciekawym wnioskiem jest iż diagram klas UML z powodzeniem nadaje się do projektowania
modelu danych, który następnie może być zaimplementowany zarówno w bazach
wykorzystujących relacyjny model danych (co autor miał już okazję robić
przy okazji innych projektów) jak również w przypadku bazy stosującej
model hierarchiczny.


% !TeX program = latexmk
% !TeX spellcheck = pl_PL
% !TeX root = example.tex

\chapter{Architektura aplikacji}

Projekt aplikacji wspomagającej zdalne szacowanie historyjek metodą Planning Poker
został wykonany w technologii, która pozwala na korzystanie z niej przy pomocy przeglądarki internetowej.

Architektura aplikacji oparta została o dwa podstawowe narzędzia:
ReactJS po stronie \textit{frontendu} oraz Firebase po stronie \textit{backendu}.
ReactJS to biblioteka napisana w języku JavaScript, stworzona przez firmę Facebook,
dzięki której budowanie dużych oraz kompleksowych interfejsów użytkownika staje się łatwiejsze.

Firebase z kolei, to usługa świadczona przez firmę Google, która
pozwala na trwałe przechowywanie danych w chmurze, oferując przy tym
wsparcie dla obustronnej komunikacji protokołem Websocket, wykorzystywanym często
przez aplikacje internetowe wymagające interakcji pomiędzy użytkownikami w czasie rzeczywistym.

\section{Czym jest ReactJS}

Twórcy ReactJS opisują go jako Widok (ang. View) w architekturze MVC\@.
Wprowadza bardzo wydajny sposób utrzymania widoku zsynchronizowanego ze stanem danych
przechowywanym w pamięci przeglądarki w postaci obiektu Javascript.

Ten specjalny stos, który renderuje HTML używa wyjątkowo szybkiego algorytmu opartego na wirtualnym drzewie DOM,
które jest ``lżejszym'' odpowiednikiem drzewa DOM\@. ReactJS wykorzystuje je
do minimalizacji liczby zmian w rzeczywistym drzewie DOM, potrzebnych do wprowadzenia go
w stan bezpośrednio wynikający z kształtu powiązanych danych.

Efektem tego algorytmu jest bardzo duża wydajność aplikacji korzystających z ReactJS oraz
łatwość pisania tych aplikacji ze względu na to, iż większością optymalizacji związanych
z renderowaniem interfejsu użytkownika zajmuje się wspomniany algorytm.

Ponadto ReactJS ma jednokierunkowy, reaktywny przepływ danych,
który jest znacznie bardziej zrozumiały i łatwiejszy w rozwoju od podejścia tradycyjnego,
czyli wielokierunkowej komunikacji pomiędzy komponentami.

Komponenty – podstawowe bloki aplikacji ReactJS – są zorganizowane w drzewie hierarchicznym,
w którym komponenty-rodzice wysyłają dane do swoich dzieci przez zmienne właściwości.
Każdy komponent ma także zmienną stanu, która determinuje obecne dane dla tego widoku.
Za każdym razem, gdy stan jest zmieniany, komponent wywołuje metodę \textit{render},
a ReactJS znajduje najbardziej efektywną metodę aktualizacji drzewa DOM\@.

Odkąd głównym zadaniem ReactJS jest interfejs użytkownika,
aplikacje na nim zrobione potrzebują czegoś jeszcze,
co będzie zachowywało się jak tzw. \textit{backend}, czyli sposób na trwałe
przetrzymywanie danych, niezależnie od stanu przeglądarki, która służy jako
środowisko uruchomieniowe dla ReactJS\@.
Do tego celu zdecydowałem się na użycie usługi Firebase.
Firebase dostarcza model (ang. Model) i kontroler (ang. Controller) w MVC
do aplikacji napisanych w ReactJS, czyniąc z nich w pełni funkcjonalne aplikacje,
których stan nie zależy wyłącznie od przeglądarki internetowej, w której działają.
Używając  jednokierunkowego systemu wiązania danych ReactJS łatwo jest zintegrować go z Firebase.
~\cite{www_react}

\section{Firebase}

Firebase jest platformą dla aplikacji webowych oraz mobilnych, która wprowadza
dla deweloperów mnóstwo narzędzi oraz usług pomagających im tworzyć wysokiej jakości
aplikacje oraz zwiększyć ich bazę użytkowników.

\subsection{Historia Firebase}

W 2011 roku Firebase znany był pod nazwą Envelope.
Jako Envelope wprowadził dla deweloperów API (Application Programming Interface),
który umożliwiał wprowadzenie komunikatora do ich strony internetowej.

Interesującym było to, że ludzie używali aplikacji by przekazywać dane,
które były czymś więcej niż wiadomościami komunikatora.
Deweloperzy aplikacji internetowych (w tym gier) używali Envelope
w celu synchronizacji danych aplikacji jak i stanu gry
w czasie rzeczywistym pomiędzy ich użytkownikami.

To doprowadziło założycieli Envelope, Jamesa Tamplina oraz Andrew Lee
do pomysłu by rozdzielić komunikator online oraz platformę wymiany danych w czasie rzeczywistym.
W kwietniu 2012 Firebase powstał jako oddzielna firma które wprowadziła usługę backendu
(ang. BaaS, Backend-as-a-Service) z funkcjonalnościami czasu rzeczywistego.

Po tym jak firma została przejęta przez Google w 2014 roku,
szybko ewoluowała do wielofunkcyjnej platformy mobilnej oraz webowej jaką znamy dzisiaj.

\section{Usługi Firebase wykorzystane w pracy}

% Rysunek
\begin{figure}
	\centering\includegraphics[width=.6\textwidth]{img/firebase.png}
	\caption{Usługi Firebase.[hackernoon.com]}\label{rys:firebase}% Źródło rysunku i etykieta przez którą odwołujemy się do rysunku.
\end{figure}

\subsection{Firebase Authentication}

W swojej aplikacji przede wszystkim skorzystałem z usługi \textit{Firebase Authentication},
która wprowadza usługi serwerowe,
łatwe narzędzia dla deweloperów oraz biblioteki Javascript
znacząco przyspieszające implementację oraz bezpieczeństwo
funkcji związanych z rejestracją oraz uwierzytelnieniem w aplikacji.

Autor skorzystał z usług logowania anonimowego oraz
tzw. \textit{third-party authentication} w usłudze GitHub, poprzez protokół OAuth.
Uwierzytelnienie poprzez usługę GitHub pozwala na integrację aplikacji z
narzędziem GitHub, co jest funkcją kluczową dla realizacji celu projektu.

\subsection{Firestore}

Firestore jest nierelacyjną bazą dokumentów, która pozwala łatwo przechowywać,
synchronizować oraz przeszukiwać dane w aplikacjach mobilnych oraz webowych – w globalnej skali.

Firestore przechowuje dane w postaci obiektów zwanych dokumentami.
Te dokumenty posiadają pary klucz-wartość oraz mogą zawierać jakiekolwiek rodzaje danych
od łańcuchów po dane binarne, a nawet obiekty, które przypominają format JSON\@.
Dokumenty są pogrupowane w kolekcje.
~\ref{rys:firestoreData}

\begin{figure}
	\centering\includegraphics[width=.6\textwidth]{img/firestoreData.png}
	\caption{Postać danych w firestore [hackernoon.com]}\label{rys:firestoreData}% Źródło rysunku i etykieta przez którą odwołujemy się do rysunku.
\end{figure}

Firestore może zawierać wiele kolekcji zawierających dokumenty, które wskazują na subkolekcje.
Te subkolekcje mogą znowu zawierać dokumenty oraz swoje subkolekcje i tak dalej.
Można zatem powiedzieć że jest to hierarchiczny model danych (patrz~\ref{rys:firestoreTree}).
~\cite{www_hakermoon}

\begin{figure}
	\centering\includegraphics[width=.6\textwidth]{img/firestoreTree.png}
	\caption{Ułożenie danych w firestore [hackernoon.com]}\label{rys:firestoreTree}% Źródło rysunku i etykieta przez którą odwołujemy się do rysunku.
\end{figure}

\section{Redux czyli implementacja architektury Flux}

Jedną z najważniejszych cech komponentów ReactJS jest wbudowany w nie stan.
Jest to bardzo przydatna koncepcja. Komponent posiada stan,
który może ulec zmianie w wyniku interakcji użytkownika z aplikacją.
Zmiana stanu pociąga za sobą operację re-renderowania drzewa Virtual DOM\@.
W wyniku tego, pewne części interfejsu widocznego na ekranie ulegają zmianie.
Oczywiście wiadomym jest też, że jeden komponent może zależeć od innego komponentu.
Możemy przecież przekazywać stan komponentu rodzica do jego komponentów dzieci itd.

To wszystko działa świetnie. Niestety w miarę jak aplikacja rośnie,
rozrasta się poziom skomplikowania poszczególnych komponentów.
Z tego względu programiści Facebook a, odpowiedzialni za rozwój ReactJS, wymyślili
architekturę aplikacji, która rozwiązuje ten problem.
Architektura ta nazywa się \textbf{Flux}.
~\cite{www_nafrontendzie}

\subsection{Architektura FLUX}

Flux jest architekturą, której Facebook używa do budowania aplikacji po stronie klienta.
Jest to uzupełnienie komponentów ReactJS przez wykorzystanie jednokierunkowego przepływu danych.
Nie jest to gotowe narzędzie, czy biblioteka, lecz pewien wzorzec architektoniczny.
Można z niego korzystać bez wielu nowych linijek kodu.

\begin{figure}
	\centering\includegraphics[width=.6\textwidth]{img/flux.png}
	\caption{Architektura Flux [wwww.nafrontendzie.pl]}\label{rys:flux}% Źródło rysunku i etykieta przez którą odwołujemy się do rysunku.
\end{figure}

Przepływ rozpoczyna się od lewej strony.
Najpierw tworzona jest akcja – jest to zwykły obiekt zawierający właściwość \texttt{type}.
Oprócz tego może on posiadać więcej właściwości służących do przekazywania dodatkowych danych.
Akcja taka tworzona jest przez funkcję zwaną \textit{action creator}, czyli kreator akcji.
W przypadku Redux, akcja jest zwykłą funkcją.

% lub {java} albo {bash} albo {text}
\begin{listing}
\begin{minted}{c}
const mapStateToProps = (state) => {
    return { counter: state.counter };
};
const mapDispatchToProps = (dispatch) => {
    return {
        onIncrement: () => dispatch({ type: 'INCREMENT' }),
        onDecrement: () => dispatch({ type: 'DECREMENT' })
    }
};

Counter = connect(mapStateToProps, mapDispatchToProps)(Counter);
\end{minted}
\caption{Przykładowe akcje licznika i ich stan} \label{listing:licznik}
\end{listing}

W listingu~\ref{listing:licznik} przedstawiony jest prosty licznik z dwoma akcjami,
które zwiększają, bądź zmniejszają licznik o jeden.
Stan oraz funkcje rozsyłające dostępne są w obiekcie \texttt{this.props}.
Spójrzmy na kod odpowiedzialny za ich mapowanie.
\begin{center}
	\textbf{Funkcja mapStateToProps}
\end{center}
Funkcja \texttt{mapStateToProps} pobiera \texttt{state} jako parametr i zwraca nowy obiekt.
Częstą praktyką jest po prostu przekazanie całego stanu do właściwości,
jednak jest to też właściwe miejsce by odfiltrować dane.
\begin{center}
	\textbf{Funkcja mapDispatchToProps}
\end{center}
Kolejna funkcja to \texttt{mapDispatchToProps}. Zwraca ona obiekt zawierający metody.
Za pomocą wywołania funkcji \texttt{dispatch} rozgłasza ona obiekty akcji do \texttt{store}.
Metoda \texttt{dispatch} przekazuje obiekty akcji bezpośrednio.

Zwykle w projekcie definiuje się to jako specjalne kreatory akcji.
Dzięki kreatorom możemy opóźnić wywołanie akcji lub zmienić dane w naszym Redux tylko wtedy,
gdy są spełnione określone warunki (patrz~\ref{listing:firebase_action}).

% lub {java} albo {bash} albo {text}
\begin{listing}
\begin{minted}{c}
export const startAddUserToGame = (owner, repo, game, user = {}) => {
    return dispatch => {
        var userUpdate = {}

        const tempUser = {
            email: user.email,
            isAnonymous: user.isAnonymous,
            id: user.uid,
            name: user.displayName,
            online: true
        };

        userUpdate[`users.` + user.uid.toString()] = tempUser
        const gameRef = db
        .collection(`users`)
        .doc(owner.toString())
        .collection(`repos`)
        .doc(repo.toString())
        .collection(`games`)
        .doc(game.toString())

        gameRef.update(userUpdate).then(() => {
            dispatch(addUserToGame(owner, repo, game, tempUser))
        });
    }
}
\end{minted}
\caption{Przykładowy kreator akcji z projektu} \label{listing:firebase_action}
\end{listing}

W tym przykładzie gracz jest dodawany do gry tylko wtedy,
gdy jest uczestnikiem gry w bazie danych Firestore.
Jak widać w przykładzie, kreator zwraca funkcję,
która wykorzystuje \texttt{dispatch} by wysłać akcję do store.
Aby zadziałać, funkcja musi znać dokładną lokalizację gry w bazie oraz dane użytkownika.
Aby dostać się do gry, musimy znać właściciela oraz nazwę repozytorium w której znajduje się gra
oraz identyfikator gry.

Dzięki funkcji \texttt{update} gracz jest dodawany do gry dostając się odpowiednio do obiektu \texttt{users},
dokumentu gry, kolekcji repozytoria, dokumentu naszego repozytorium.
Kolekcja gier znajduje się w repozytorium, a dzięki znajomości identyfikatora gry
dostajemy się do jej obiektu i wstawiamy do niej obiekt gracza.
Jeżeli akcja w bazie danych zakończy się sukcesem, rozsyłamy akcję dodawania gracza do store.

\begin{center}
	\textbf{Funkcja connect}
\end{center}
Wracając do mapowania. W przedstawionym wcześniej przykładzie najbardziej istotna jest ostatnia jego linia.
Wywołuje ona funkcję \texttt{connect}.
Przyjmuje ona funkcje \texttt{mapStateToProps} oraz \texttt{mapDispatchToProps}
jako parametry i wyniki ich wywołania łączy w odpowiedni obiekt.
Następnie zwraca funkcję, która jako parametr przyjmuje komponent.
Funkcja ta wprowadza przygotowany wcześniej obiekt do \texttt{this.props} tego komponentu.

Funkcja connect opakowuje przekazany komponent i zwraca nową jego wersję.

Wracając do Flux. Tak tworzona akcja jest dostarczana do store za pomocą wywołania funkcji \texttt{dispatcher}.
Funkcja ta w zasadzie zarządza całym przepływem danych.
Każdy store w aplikacji rejestruje w dispatcher swoje funkcje wywołania zwrotnego
w celu obsługi przychodzących akcji.
W momencie gdy akcja jest rozsyłana (ang.\ dispatch), wywoływane są po kolei wszystkie
funkcje wywołania zwrotnego (ang.\ callback).
Jeden z nich powinien umieć rozpoznać akcję po jej typie i być przygotowany na jej odpowiednią obsługę.

\begin{center}
	\textbf{Funkcja reducer}
\end{center}

W przypadku Redux rolę store pełni funkcja \textit{reducer}. Dzięki temu, że jest to funkcja,
można ją jednocześnie użyć jako funkcję wywołania zwrotnego,
którą store uruchomi w momencie gdy zostanie rozgłoszona jakaś akcja.
Funkcja ta przyjmuje dwa parametry: \texttt{state} oraz \texttt{action}.

Działa to tak: ktoś wywołuje akcję, obiekt \textbf{store} wywołuje funkcję \textbf{reducer},
przekazując do niej aktualny stan oraz akcję, \textbf{reducer}
sprawdza typ przekazanej do niej akcji i w zależności od tego jaki jest ten typ,
zwraca nową wersję obiektu stanu (patrz przykład~\ref{listing:reducer}).

\begin{listing}
\begin{minted}{c}
const reducer = (state, action) => {
    switch (action.type) {
        case 'INCREMENT':
            return { ...state, counter: state.counter + 1 };
        case 'DECREMENT':
            return { ...state, counter: state.counter - 1 };
        default:
            return state;
    }
};
\end{minted}
\caption{Przykładowy reducer licznika} \label{listing:reducer}
\end{listing}

W przykładzie, jeśli typ akcji to `INCREMENT' to zwracany jest nowy obiekt stanu,
który ma zwiększoną o jeden wartość atrybutu \texttt{counter}.
Jeśli natomiast typ akcji to `DECREMENT' to zwracany jest nowy stan z atrybutem
\texttt{counter} zmniejszonym o jeden.
Jeśli typ akcji jest w tym miejscu nieznany, wykonujemy domyślną (ang.\ default) operację, którą jest
przekazanie obiekt stanu w niezmienionej postaci.

Generalnie wszystkie obiekty store zawierają łącznie cały stan aplikacji.
Store zawiera implementację funkcji wywoływania zwrotnego,
która jest rejestrowana w ``dispatcher'', i która obsługuje akcje związane z danym store.
Cechą Redux jest to, że cały stan jest przechowywany w jednym obiekcie.

Kolejny element na diagramie to widok.
Można powiedzieć, że jest on reprezentowany po prostu przez komponent ReactJS\@.
Używa on stanu aplikacji zapisanego w obiekcie store, tak jakby był on wewnętrznym stanem komponentu.
Zmiana stanu w store powoduje re-renderowanie komponentu.
Dodatkowo komponent widoku może rozsyłać kolejne akcje, na przykład kiedy użytkownik
kliknie na jakiś przycisk interfejsu na ekranie.
To powoduje zmianę stanu zapisanego w store.
W Redux komponent ma dostęp do akcji i stanu dzięki wspomnianej przeze mnie wcześniej funkcji \texttt{connect}.

Wszystko to po prostu zestaw zasad.
To programista może zdecydować, jak to dokładnie będzie zaimplementowane.
Na szczęście nie jest on zdany sam na siebie, ponieważ istnieje kilka gotowych
implementacji architektury Flux w postaci bibliotek.
Jedną z nich jest oczywiście Redux.
~\cite{www_nafrontendzie}

\subsection{Czym Redux różni się od Flux}

Redux jest inspirowany pewnymi ważnymi cechami architektury Flux.
Tak jak Flux, Redux przepisuje model danych do oddzielnej warstwy aplikacji
(\textit{stores} w Flux, \textit{reducers} w Redux).

Tak jak we Flux, wszystkie operacje na danych są opisywane w postaci akcji.
W przeciwieństwie do Flux, Redux jednak nie ma konceptu rozsyłacza,
ponieważ opiera się na czystych funkcjach zamiast obiektów emiterów akcji,
a czyste funkcje są łatwe do tworzenia i nie potrzebują żadnej encji
(reprezentacja wyobrażonego lub rzeczywistego obiektu) by nimi zarządzać.
Jest to więc pewne odstępstwo od Flux.
~\cite{www_nafrontendzie}

Inną ważną różnicą od Flux jest to, że Redux zakłada niemutowalność danych.
Możesz używać statycznych obiektów oraz tablic dla swojego stanu,
ale mutowanie ich wewnątrz reducerów jest bardzo odradzane,
dlatego po zmianie stanu zawsze powinien być zwracany nowy obiekt (patrz rysunek~\ref{rys:reduxFlux}).

Ogólnie Redux może być opisany przez trzy fundamentalne zasady:
\begin{itemize}
	\item Pojedyncze źródło prawdy: stan całej aplikacji przetrzymywany jest
	w drzewie obiektów wewnątrz pojedynczego obiektu store.
	\item Stan jest tylko do odczytu: jedynym sposobem na zmianę stanu jest wywołanie akcji,
	która zwraca obiekt opisujący co powinno się stać.
	\item Zmiany wykonywane są w ramach czystych funkcji: aby określić jak drzewo stanu
	transformowane jest przez akcje, musisz tworzyć ``czyste reducery''.
\end{itemize}

\begin{figure}
	\centering\includegraphics[width=\textwidth]{img/reduxFlux.jpeg}
	\caption{Różnice między Redux a Flux [www.medium.com]}\label{rys:reduxFlux}% Źródło rysunku i etykieta przez którą odwołujemy się do rysunku.
\end{figure}

\subsection{redux-thunk}

Biblioteka \textbf{redux-thunk} została stworzona przez Dana Abramova,
który jednocześnie jest twórcą Redux. Pozwala ona tworzyć kreatory akcji,
które zamiast obiektu zwracają funkcję.
Dzięki temu możliwe jest opóźnienie rozgłoszenia akcji lub zgłoszenie jej tylko
jeśli zostaną spełnione określone warunki.~\cite{www_thunk}

Jak dodać thunk do Redux pokazuje listing~\ref{listing:thunk-redux}.

\begin{listing}
\begin{minted}{c}
    import { createStore, applyMiddleware } from 'redux';
    import thunk from 'redux-thunk';
    import rootReducer from './reducers/index';

    const store = createStore(
        rootReducer,
        applyMiddleware(thunk)
    );
\end{minted}
\caption{Połączenie Redux i Thunk} \label{listing:thunk-redux}
\end{listing}

\begin{center}
	\textbf{Obsługa wywołań asynchronicznych czyli kreatory akcji}
\end{center}

W przykładzie (listing~\ref{listing:firebase_action}) widnieje akcja,
która najpierw dodaje użytkownika do bazy, a następnie rozgłasza akcję dodawania użytkownika,
jeżeli modyfikacja danych w bazie zakończyła się pomyślnie.

W funkcji \texttt{startAddUserToGame} pobierane są najpierw dane o grze, czyli: \texttt{owner},
które są po prostu identyfikatorem użytkownika GitHub.
Następnie jest pobierany identyfikator gry \texttt{game}.
Na końcu dostajemy dane użytkownika, którego chcemy dodać do gry.

Funkcja ta zwraca funkcję przyjmującą parametr \texttt{dispatch},
który służy do rozgłaszania danych.
Później jest tworzony użytkownik, który ma zostać dodany do gry (\texttt{tempUser}).
Na końcu jest tworzony obiekt, będący odnośnikiem do użytkowników w bazie,
który pomaga zaktualizować obiekt gry.
Finalnie korzystając z referencji aktualizuje się obiekt gry w bazie danych,
a w razie sukcesu (funkcja \textbf{then}) akcja jest rozsyłana dalej.

\subsection{Połączenie Firebase i Redux}

Jak nie trudno się domyślić, wszystkie operacje pomiędzy Firebase a Redux są realizowane w postaci akcji.
Do synchronizacji frontend i backend służy \textit{Thunk},
który przesyła dane do store tylko wtedy,
kiedy nie ma żadnych problemów po stronie Firebase.

Jeżeli baza danych nie będzie mogła pobrać danych bo wystąpił błąd,
Thunk nie wykona żadnych operacji. Sposób połączenia tych dwóch elementów przedstawiono
na rysunku~\ref{rys:fireRedux}.

\begin{figure}
	\centering\includegraphics[width=.6\textwidth]{img/fireRedux.png}
	\caption{Sposób połączenia firebase i reduxa. [medium.com]}\label{rys:fireRedux}% Źródło rysunku i etykieta przez którą odwołujemy się do rysunku.
\end{figure}



% !TeX program = latexmk
% !TeX spellcheck = pl_PL
% !TeX root = example.tex

\chapter{Implementacja projektu}

Aplikacja \textit{Planning Poker}, to aplikacja webowa oparta na platformie Firebase,
która jest napisana przy użyciu biblioteki ReactJS, z wykorzystaniem
edytora Visual Studio Code.

Aby zachować porządek i możliwość późniejszego rozwoju aplikacji o nowe komponenty,
koniecznym jest stworzenie uporządkowanej struktury plików zawartych w projekcie
(rysunek~\ref{rys:projekt}).

\begin{figure}[h]
	\centering\includegraphics[width=.3\textwidth]{img/projekt.png}
	\caption{Struktura projektu}\label{rys:projekt}% Źródło rysunku i etykieta przez którą odwołujemy się do rysunku.
\end{figure}

Folderem zawierającym wszystkie podfoldery oraz pliki jest folder \texttt{scrumpoker}.
Pliki takie jak \texttt{storage.rules} czy \texttt{.firebaesrc} powstały automatycznie
w procesie konfiguracji usługi Firebase.

Plik \texttt{package.json} zawiera całą konfigurację projektu stworzonego
za pomocą narzędzia \texttt{create-react-app}.

Folder \texttt{node modules} zawiera zależności, niezbędne do uruchomienia aplikacji.

Folder \texttt{src} zawiera cały kod źródłowy aplikacji.
Głównym plikiem projektu jest \texttt{index.jsx}. To w tym pliku renderowana jest cała aplikacja.

Aby uruchomić program,
niezbędny jest też backend w postaci utworzonego i skonfigurowanego projektu Firebase.
Konfiguracje należy umieścić w pliku \texttt{src/firebase/firebase.jsx}
(patrz listing~\ref{listing:firebaseConfig}).

\begin{listing}
\begin{minted}{c}
import * as firebase from 'firebase';

var config = {
    apiKey: "<API_KEY>",
    authDomain: "<PROJECT_ID>.firebaseapp.com",
    databaseURL: "https://<DATABASE_NAME>.firebaseio.com",
    projectId: "<PROJECT_ID>",
    storageBucket: "<BUCKET>.appspot.com",
    messagingSenderId: "<SENDER_ID>",
};

firebase.initializeApp(config);

const database = firebase.firestore();
const settings = { timestampsInSnapshots: true};
database.settings(settings);
export { firebase, database as default };
\end{minted}
\caption{Konfiguracja firebase}\label{listing:firebaseConfig}
\end{listing}

Z wyjątkiem stanu gry, aplikacja wszelkie dane pobiera z serwisu Github poprzez
udostępnione API, wykorzystując do tego biliotekę \texttt{octokit/rest}.

W każdym komponencie, wymagającym danych z tego serwisu,
uruchamiane są metody, które ładują niezbędne dane do stanu komponentu.

Jako że wszelkie gry tworzone w aplikacji dotyczą projektów prowadzonych na Github'ie,
użytkownik, aby skorzystać z aplikacji, musi posiadać konto na Github'ie oraz co najmniej
jeden projekt założony w tej usłudze.
Biorąc pod uwagę, że w trakcie rozgrywki Planning Pokera oceniane są historyjki z backlogu,
użytkownik musi mieć jakieś historyjki przypisane do wybranego projektu w Github.
Aby to zrobić musi utworzyć w projekcie, elementy zwane `issues', które w tym momencie traktowane są jako historyjki.
Co za tym idzie, użytkownik musi się zalogować przez serwis Github, co ilustruje poniższy rysunek~\ref{rys:login}.

\begin{figure}[h]
	\centering\includegraphics[width=\textwidth]{img/GitLogin.png}
	\caption{Strona logowania}\label{rys:login}% Źródło rysunku i etykieta przez którą odwołujemy się do rysunku.
\end{figure}

\section{Tworzenie gry}

Można założyć, że grupą docelową aplikacji, są użytkownicy Githuba, czyli programiści.
Autor, jako częsty użytkownik tej usługi, odczuł brak aplikacji wspomagającej Planning Poker,
która byłaby z tym narzędziem zintegrowana.

\begin{figure}[h]
	\centering\includegraphics[width=\textwidth]{img/repositories.png}
	\caption{Ekran wyboru projektu}\label{rys:projekty}% Źródło rysunku i etykieta przez którą odwołujemy się do rysunku.
\end{figure}

Autor podczas przeglądania projektów na Github'ie zauważył, że jego użytkownicy,
wszelkie problemy czy też propozycje nowych funkcjonalności aplikacji umieszczają
w sekcji \textbf{issues} danego projektu. Stwierdził wówczas, iż aplikacja która ocenia
istniejące problemy lub proponowane cechy, które są dostępne w już sprawdzonym narzędziu
i eksportuje oceny do Github'a w postaci etykiet będzie dobrym pomysłem.


\subsection{Ekran wyboru projektu}

Aby utworzyć grę, najpierw należy wybrać projekt, którego gra będzie dotyczyła.
W aplikacji projektami są repozytoria (rysunek~\ref{rys:projekty}).

\begin{figure}[h]
	\centering\includegraphics[width=\textwidth]{img/Issues.png}
	\caption{Ekran wyboru backlogu}\label{rys:issues}% Źródło rysunku i etykieta przez którą odwołujemy się do rysunku.
\end{figure}


\subsection{Ekran wyboru backlogu}

Aby rozgrywka miała sens, konieczne jest posiadanie backlogu zadań,
które zespół powinien wyestymować podczas danej sesji planingowej.
Odpowiednią listę zadań Tworzy się ją w panelu ustawień projektu (rysunek~\ref{rys:issues}),
w sekcji \textbf{Issues}.

Dzięki odpowiednim ustawieniom filtrów można odnaleźć elementy po etykietach,
autorze czy po kamieniach milowych.

Aby stworzyć listę, należy wybrać odpowiednie zagadnienia, klikając lewy przycisk myszy,
czy też zaznaczając stronę za pomocą przycisku \textbf{Check page}.

Po wybraniu odpowiednich elementów i naciśnięciu przycisku \textbf{Create List}
pokazywane jest okno potwierdzenia,
które pokazuje wybrane elementy oraz pole do wpisania nazwy listy.

\begin{figure}[h]
	\centering\includegraphics[width=0.8\textwidth]{img/formularz.png}
	\caption{Formularz tworzenia gry}\label{rys:form}% Źródło rysunku i etykieta przez którą odwołujemy się do rysunku.
\end{figure}

Po potwierdzeniu wyboru oraz wpisaniu nazwy listy, jest tworzona lista, zawierająca
identyfikatory wybranych historyjek (czyli \textit{issues} w projekcie Github).

Jak zostało już wcześniej wspomniane, niezbędnym elementem gry jest backlog.
Tym backlogiem dla każdej gry będzie lista, którą właściciel tworzy spośród wspomnianych historyjek.
Ta lista będzie później dostępna w formularzu tworzenia gry,
dzięki czemu będzie można ją wybrać przy jej tworzeniu.


\subsection{Ekran ustawień gry}

W trakcie tworzenia gry możemy skorzystać z ekranu ustawień gry (rysunek~\ref{rys:form})
aby określić jej nazwę i opis ale także parametry, które będą miały wpływ na grę.
Oczywiście, będziemy mogli wybrać skalę estymacji gry, ale także podać prędkość (ang.\ velocity)
naszego zespołu, zdecydować czy inni gracze będą tą prędkość widzieć.
Możemy także zdecydować czy twórca gry również będzie mógł głosować.

Inną ważną kwestią jest to, czy karty powinny być odkrywane, kiedy każdy zagłosował
oraz czy pozwolić graczom zmienić swoją ocenę po pokazaniu wszystkich kart.

Aplikacja może również obliczać wynik punktowy historyjki po zakończonym głosowaniu automatycznie,
robi to za pomocą średniej ważonej, która później jest zamieniana na najbliższą jej ocenę w skali estymacji.

Na końcu oczywiście należy wybrać backlog w postaci odpowiedniej listy, tu należy zaznaczyć,
że jeżeli w projekcie nie będzie stworzonej żadnej listy, to nie będzie można stworzyć gry.

Pewne ustawienia gry można zmienić także już po jej utworzeniu.

\section{Rozgrywka}

Do rozgrywki Planning Pokera oczywiście potrzebni są gracze.
Aby rozgrywka była możliwa dla więcej niż jednego gracza, konieczna jest komunikacja z serwerem.
Wszelkie akcje podczas gry są synchronizowane za pomocą Firebase,
a dzięki odpowiednim słuchaczom aplikacja reaguje na wszelkie zmiany w stanie gry.

Gracz jest dodawany do gry poprzez udostępnianie innym użytkownikom odnośnika do gry
w postaci jej adresu URL\@.
Jeżeli gracz wejdzie do gry, zostanie poproszony o podanie nazwy,
dzięki której będzie można go odróżnić od innych graczy.
Gracze są dodawani do bazy danych dzięki funkcji usługi Firebase nazwanej anonimowe logowanie.
Gracze w grze mogą tylko i wyłącznie głosować,
natomiast właściciel gry wybiera historyjkę do głosowania, to znaczy że to on narzuca tempo gry.
Panel gry pokazuje rysunek~\ref{rys:gra}.

\begin{figure}[h]
	\centering\includegraphics[width=\textwidth]{img/gra.png}
	\caption{Panel gry}\label{rys:gra}% Źródło rysunku i etykieta przez którą odwołujemy się do rysunku.
\end{figure}

\section{Komunikacja słowna}

Istotnym elementem każdego Planning Pokera jest komunikacja między graczami,
autor jednak stwierdził,
iż na rynku istnieją już odpowiednie aplikacje do komunikacji,
a firmy produkujące oprogramowanie mają preferowane komunikatory, których używają
również w innych sytuacjach niż proces planowania sprintu.
A więc przed każdą rozgrywką autor poleca wybrać odpowiedni komunikator do komunikacji w trakcie gry.


\section{Wynik implementacji}

Nie wszystkie zaplanowane elementy znajdują się w aplikacji,
jednak najważniejsze funkcjonalności zostały zaimplementowane:

 \begin{itemize}
	\item Logowanie przez Github'a
	\item Pobieranie Issues z Github'a
	\item Tworzenie list, które będą backlog'iem w grze
	\item Tworzenie gier
	\item Rozgrywka
	\item Eksport wyników gier do Github'a
	\item Możliwość usunięcia gry
	\item Możliwość zmiany ustawień gry
	\item Możliwość zapraszania użytkowników do gry
\end{itemize}

Funkcje których nie udało się zaimplementować to:

\begin{itemize}
	\item Pełna synchronizacja z Github'em
	\item Edycja historyjek wraz z synchronizacją zmian z Github'em
	\item Możliwość usuwania użytkownika z gry
\end{itemize}

\section{Podsumowanie}

Dzięki użyciu odpowiednich technologii udało się stworzyć w pełni funkcjonalną aplikację,
umożliwiającą zespołom zdalnym przeprowadzenie sesji planningowej nad dobrze
zdefioniowanym backlogiem zadań.
Nie znaczy to jednak, że jej stworzenie było proste.
Autorowi zajęło trochę czasu zanim wybrał odpowiednie technologie do projektu.
Można powiedzieć że niniejsza praca była świetną okazją do poznania
wielu nowych narzędzi do tworzenia aplikacji internetowych wykorzystujących
komunikację w czasie rzeczywistym.


\chapter{Ewaluacja aplikacji}

\section{Cel}
Celem badania jest poznanie zdania prawdziwych użytkowników.
\section{Plan badania}
Aplikacja zostanie przedstawiona zespołowi,
który tworzy oprogramowanie z wykorzystaniem metodyk zwinnych.
Jako że członkowie zespołu po usłyszeniu założeń projektu, stwierdzili,
że nie chcą aby aplikacja ewentualnia zniszczyła któryś z ich projektów,
a sami nie chcą tworzyć nowych historyjek stwierdziliśmy,
że w ramach testu stworzę testowe issues.
Dlatego właśnie autor aplikacji będzie scrum masterem a badani będą graczami,
więc będą oceniać historyjki.
Jako że aplikacja jest podobna do funkcjonujących na rynku,
nie trzeba było zbytnio badanym objaśnień działania aplikacji.
Badanie będzie wykonane przy skali Fibunacciego,
karty będą automatycznie odkrywane gdy wszyscy zagłosowali.
 Aplikacja będzie obliczała sama końcową ocenę a scrum master nie będzie głosował.
 \textbf{Nazwa badania}: Planning poker
 \textbf{Backlog}: Saving games to database, Filter issues, Adding list to database,
 Implementation of gameplay, Interface for game, connecting planning poker to firebase,
implementation of game cards, adding create game form, integration with github.
\section{Wyniki}
Przebadani użytkownicy mieli pod przewodnictwem autora ocenić 10 testowych historyjek.
Mieli oni korzystać z czterech różnych przeglądarek.
Badanie trwało 10 min. po których użytkownicy mieli wypełnić ankietę i dodać ewentualne uwagi.
~\ref{rys:wynikBadania}
\begin{figure}[H]
	\centering\includegraphics[width=.9\textwidth]{img/wynikBadania}
	\caption{Ocenione historyjki w Github'ie}.
	\label{rys:wynikBadania}
\end{figure}



\chapter*{Zakończenie}

Celem pracy było zaprojektowanie oraz zaimplementowania narzędzia wspomagającego szacowanie zadań metodą planning pokera.
Udało się zrealizować założenia projektu, jednak koncepcja często się zmieniała,
co wynikało z braku doświadczenia w tworzeniu aplikacji czasu rzeczywistego oraz poszukiwania odpowiednich rozwiązań.
Jednak aplikacja dzięki wykorzystaniu odpowiednich narzędzi została stworzona i przetestowana.
Uważam że usługa firebase świetnie się nadaje do aplikacji wymagających asynchronicznej komunikacji.
Aplikacja została przetestowana na wszystkich przeglądarkach, a nawet na urządzeniach mobilnych.
Dzięki usłudze firebase hosting jest ona też dostępna w internecie.
Jako że podobne aplikacje już istnieją,
sądze że przedstawiony w niniejszej pracy program może znaleźć zainteresowanie wśród programistów.
Sądząc po trendach na rynku pracy, aplikację które pomagają w pracy zdalnej będą coraz bardziej popularne.
Co prawda autor nie sądzi by jego aplikacja zdobyła dużą popularność,
 jednak tego typu narzędzia będą coraz bardziej popularne dla wielu dzisiejszych i przyszłych firm.

\appendixpage
\appendix
%\addappheadtotoc

\bibliography{literatura}
\bibliographystyle{dyplom}

\end{document}

