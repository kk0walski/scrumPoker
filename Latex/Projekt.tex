\chapter{Projekt aplikacji}
W niniejszym rozdziale autor umieścił projekt aplikacji oraz bazy danych.
\section{Diagram przypadków użycia}
\begin{figure}[H]
	\centering\includegraphics[width=0.98\textwidth]{img/UseCase.pdf}
	\caption{Diagram przypadków użycia}.\label{rys:useCase}
\end{figure}
\textbf{Product owner chce się zalogować}
\begin{enumerate}
    \item System prosi użytkownika o zalogowanie.
    \item Użytkownik naciska przycisk.
    \item Dostawca usługi logowania loguje użytkownika lub prosi o dane logowana.
    \item Product owner\footnote{Product owner jest moderatorem gry oraz osobą z dostępem do repozytorium projektu} zostaje zalogowany do systemu.
    \item Product owner przegląda listę swoich projektów.
\end{enumerate}
\textbf{Product owner chce stworzyć listę}
\begin{enumerate}
    \item Product owner loguje się do systemu.
    \item Product owner wybiera projekt dla którego chce stworzyć listę.
    \item Product owner wybiera elementy do nowej listy z `Issues' Github'a.
    \item Product owner naciska przycisk do tworzenia listy.
    \item Product owner nadaje listę nazwę i potwierdza wybór historyjek.
\end{enumerate}
\textbf{Product owner chce stworzyć grę}
\begin{enumerate}
    \item Product owner loguje się do systemu.
    \item Product owner wybiera projekt dla którego chce stworzyć grę.
    \item Jeżeli w projekcie jest stworzona jakakolwiek lista system prosi product ownera o wypełnienie formularza,
    w przeciwnym razie system odsyła użytkownika do listy issue's GitHub w celu stworzenia listy.
    \item Użytkownik potwierdza swoje ustawienia.
    \item System tworzy grę i przekierowuje użytkownika do listy gier projektu.
\end{enumerate}
\textbf{Product owner chce przeprowadzić grę}
\begin{enumerate}
    \item Product owner loguje się do systemu.
    \item Product owner tworzy grę.
    \item Product owner wchodzi do gry.
    \item Product owner rozsyła link do gry graczom.
    \item Gracze głosują nad historyjką.
    \item Product owner zmienia historyjki nadając tempo grze.
\end{enumerate}
\textbf{Product owner chce zmienić ustawienia gry}
\begin{enumerate}
    \item Product owner loguje się do systemu.
    \item Product owner wybiera projekt.
    \item Product owner wybiera grę.
    \item Product owner wchodzi do gry.
    \item Product owner naciska przycisk ustawienia.
    \item Product owner przechodzi do ekranu ustawień gry.
    \item Product owner zmienia ustawienia.
    \item Product owner zatwierdza ustawienia.
    \item System zapisuje ustawienia w bazie.
\end{enumerate}
\textbf{Product owner chce usunąć gracza z gry}
\begin{enumerate}
    \item Product owner loguje się do systemu.
    \item Product owner wybiera projekt.
    \item Product owner wybiera grę.
    \item Product owner wchodzi do gry.
    \item Product owner naciska przycisk `Gracze'.
    \item System wyświetla product ownerowi listę graczy.
    \item Product owner naciska przycisk usunięcia przy nazwie gracza.
    \item System usuwa gracza oraz wszystkie jego oceny z gry ponownie obliczając punktację.
\end{enumerate}
\textbf{Product owner chce wyeksportować punktację}
\begin{enumerate}
    \item Product owner loguje się do systemu.
    \item Product owner wybiera projekt.
    \item Product owner wybiera grę.
    \item Product owner naciska przycisk eksport obok przycisku usuwania gry.
    \item System wyświetla product ownerowi listę historyjek wraz z ich punktacją.
    \item Product owner zatwierdza punktację.
    \item System przenosi punktację historyjek w postaci etykiet do Github'a.
\end{enumerate}
\textbf{Product owner chce usunąć grę}
\begin{enumerate}
    \item Product owner loguje się do systemu.
    \item Product owner wybiera projekt.
    \item Product owner wybiera grę.
    \item Product owner naciska przycisk usuń.
    \item System usuwa grę z bazy.
    \item Gra znika z listy gier.
\end{enumerate}
\section{Interfejs}
Aby ułatwić implementację systemu, autor zamodelował ekrany aplikacji w postaci
tzw.\ makiet (ang.\ mockups):
\begin{itemize}
    \item Ekranu logowania
    \item Ekrany listy projektów
    \item Ekrany projektu wraz z bocznym menu (ekran przedstawiający issues)
    \item Ekran formularza tworzenia gry
    \item Ekran listy gier
    \item Ekran gry
\end{itemize}
Prototyp interfejsu użytkownika został stworzony dzięki narzędziu Visual Paradigm.
\begin{figure}[H]
	\centering\includegraphics[width=.7\textwidth]{img/LoginScreen.png}
	\caption{Ekran Logowania}.\label{rys:loginScreen}
\end{figure}
\begin{figure}[H]
	\centering\includegraphics[width=.7\textwidth]{img/RepositoriesScreen.png}
	\caption{Ekran główny aplikacji z repozytoriami}.\label{rys:RepositoriesScreen}
\end{figure}
\begin{figure}[H]
	\centering\includegraphics[width=.7\textwidth]{img/IssuesScreen.png}
	\caption{Główny ekran projektu}.\label{rys:IssuesScreen}
\end{figure}
\begin{figure}[H]
	\centering\includegraphics[width=.7\textwidth]{img/gameCreate.png}
	\caption{Formularz tworzenia gry}.\label{rys:gameCreate}
\end{figure}
\begin{figure}[H]
	\centering\includegraphics[width=.7\textwidth]{img/GamesList.png}
	\caption{Ekran gier}.\label{rys:GamesList}
\end{figure}
\begin{figure}[H]
	\centering\includegraphics[width=.7\textwidth]{img/GameScreen.png}
	\caption{Panel gry}.\label{rys:GameScreen}
\end{figure}

Na rysunku:~\ref{rys:ScreensDiagram} przedstawiono diagram przepływu pomiędzy
zaprezentowanymi powyżej makietami ekranów aplikacji.

\begin{figure}[H]
	\centering\includegraphics[width=\textwidth]{img/ScreensDiagram.png}
	\caption{Diagram przepływu między ekranami aplikacji}.\label{rys:ScreensDiagram}
\end{figure}

\section{Baza danych}

\begin{figure}[H]
	\centering\includegraphics[width=\textwidth]{img/ClassDiagram.png}
	\caption{Diagram klas, przedstawiający koncepcję bazy danych}.\label{rys:ClassDiagram}
\end{figure}

Na rysunku~\ref{rys:ClassDiagram} przedstawiono model danych w aplikacji.
W tym miejscu wartym odnotowania jest fakt, iż autor zdecydował się
na zastosowanie hierarchicznej bazy danych Firestore, zamiast którejś z najczęściej
wybieranych relacyjnych baz danych.
Więcej o Firestore znajdziemy w rozdziale o architekturze aplikacji.

Ciekawym wnioskiem jest iż diagram klas UML z powodzeniem nadaje się do projektowania
modelu danych, który następnie może być zaimplementowany zarówno w bazach
wykorzystujących relacyjny model danych (co autor miał już okazję robić
przy okazji innych projektów) jak również w przypadku bazy stosującej
model hierarchiczny.
