% !TeX program = latexmk
% !TeX spellcheck = pl_PL
% !TeX root = example.tex

\chapter{Rozproszone zespoły projektowe}

Według niemieckiego pisma \textit{Focus Money Magazin},
ok 30 \% wszystkich pracowników na świecie (a podobno w USA ponad 43 \%) pracuje w tzw.
zespołach rozproszonych lub wirtualnych.
Wiele firm wprowadza zarządzanie na odległość, aby w ten sposób stymulować rozwój,
rozwijać projekty międzynarodowe, czy chociażby ze względu na oszczędności lub optymalizację procesów.

Pojawienie się zjawiska zespołów rozproszonych może się wiązać z efektem globalizacji obszaru działania firm,
wkraczaniem na nowe rynki, powstawaniem \textit{startupów}, jak również działaniem tzw free-lancerów,
którzy oferują wyspecjalizowane usługi.
Powodem mogą być również wzrost kosztów pracy, zmiany na rynku pracy oraz brak na rynku lokalnym
wykwalifikowanych pracowników, jak również zmiany w kulturze pracy
(np. pokolenie tzw. Millenialsów). Wszystko to powoduje zmiany w organizacji pracy.

\section{Zalety zarządzania projektami w zespołach rozproszonych}

Realizacja projektów zespołami rozproszonymi:

\begin{itemize}
	\item pozwala na pozyskiwanie profesjonalistów w ramach całej organizacji lub poza nią;
	\item niweluje problem ograniczeń korzystania wyłącznie z zasobów jednego biura, czy w ramach jednego kraju;
	\item daje szerokie możliwości czerpania z dużo większych zbiorów doświadczeń oraz kreatywności;
	\item pozwala na elastyczność w rozbudowywaniu zespołu adekwatnie do pojawiających się nowych zadań;
	\item daje organizacjom szansę dostosowywania się do zmian pokoleniowych,
	gdzie często młode osoby wyrażają potrzebę wykonywania pracy w domu (ang. Home Office);
	\item zwiększa efektywność kosztową.
\end{itemize}

\section{Dodatkowe wyzwania dla efektywnego funkcjonowania zespołów rozproszonych}

Każdy zespół projektowy napotyka w swojej pracy na najróżniejsze wyzwania,
które związane są ze współpracą, komunikacją, czasami niejasnym podziałem obowiązków,
brakiem odpowiedzialności, czy zaangażowaniem.
Jednak zespoły rozproszone stają często przed dodatkowymi wyzwaniami, jakie wynikają z ich specyfiki,
czyli pracy w różnych lokalizacjach.

Oto kilka najczęstszych wyzwań takich zespołów:

\begin{itemize}
	\item \textbf{Zarzadzanie zespołem i synchronizacja pracy jego członków}

	To wyzwanie nie tylko dotyczy zespołów rozproszonych,
	ale w właśnie w takich zespołach szczególnie ważna jest umiejętność prowadzenia projektu
	i kwalifikacje związane z zarządzaniem ludźmi.
	W zespole rozproszonym powinny funkcjonować jasne zasady współpracy i komunikacji.
	Bardzo istotne jest ogólne zaangażowanie, wzajemne rozliczanie się z zadań,
	odpowiedzialność za rezultaty, otwartość, szacunek dla innych,
	zakładanie zawsze dobrych intencji, jasny podział obowiązków.
	Ważne jest również, dla budowania zespołu i jego efektywności, aby zaplanować
	(w budżecie oraz czasowo) okresowych, wspólnych spotkań wszystkich członków.

	\item \textbf{Przepływ informacji}

	Czasem kluczowe informacje nie docierają do wszystkich członków zespołu.
	Odpowiedzią na to może być ustalenie pewnego ``rytmu'' komunikacyjnego i regularne spotkania projektowe.
	Muszą to jednak być spotkania z sensem, tak zaplanowane, aby nie marnować czasu pracowników,
	aby rozwiązywały konkretne problemy i wspierały realizację celów projektu.
	Sposobów komunikacji jest wiele, można zatem je dobrze zaplanować i stosować,
	w zależności od złożoności zespołu, celu i rodzaju projektu.

	\item \textbf{Komunikacja w obcym języku}

	Coraz powszechniejsza jest konieczność porozumiewania się w języku obcym,
	ponieważ wiele firm posiada zagraniczne filie i siedziby.
	Zazwyczaj wykorzystywany do komunikacji jest język angielski,
	ale poziom jego znajomości może być różny i może powodować różne zabawne sytuacje lub nawet nieporozumienia.
	Tutaj może pomóc np. nauka prostych technik coachingowych - pytań doprecyzowujących i tzw. uważne słuchanie.
\end{itemize}

Dodatkowo zespoły rozproszone muszą się borykać z problemami takimi jak: różnica stref czasowych,
różnice kulturowe, odpowiednia rekrutacja, motywowanie i wsparcie zespołu, efektywność i produktywność,
zwiększa efektywność kosztową.

Dochodzimy wreszcie do kwestii, której bezpośrednio dotyczy temat niniejszej pracy inżynierskiej.
Jest to wyzwanie:

\begin{itemize}
	\item \textbf{Wsparcie techniczne i merytoryczne}

	W zespołach rozproszonych wręcz niezbędne jest zadbanie o wsparcie techniczne.
	Co więcej, wielu menedżerów zarządzających zespołami rozproszonymi,
	wskazuje na potrzebę wsparcia merytorycznego,
	szczególnie w zakresie moderowania tzw. metodyk zwinnych (SCRUM).
	Świetne połączenie internetowe,	sprzęt, sale i programy do tele- i wideokonferencji,
	to powinien być standard w organizacjach, które pracują w zespołach wirualnych.
\end{itemize}


Więcej informacji o zespołach rozproszonych można znaleźć w \cite{www_rozproszony}.
